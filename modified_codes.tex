% Options for packages loaded elsewhere
\PassOptionsToPackage{unicode}{hyperref}
\PassOptionsToPackage{hyphens}{url}
%
\documentclass[
]{article}
\usepackage{amsmath,amssymb}
\usepackage{iftex}
\ifPDFTeX
  \usepackage[T1]{fontenc}
  \usepackage[utf8]{inputenc}
  \usepackage{textcomp} % provide euro and other symbols
\else % if luatex or xetex
  \usepackage{unicode-math} % this also loads fontspec
  \defaultfontfeatures{Scale=MatchLowercase}
  \defaultfontfeatures[\rmfamily]{Ligatures=TeX,Scale=1}
\fi
\usepackage{lmodern}
\ifPDFTeX\else
  % xetex/luatex font selection
\fi
% Use upquote if available, for straight quotes in verbatim environments
\IfFileExists{upquote.sty}{\usepackage{upquote}}{}
\IfFileExists{microtype.sty}{% use microtype if available
  \usepackage[]{microtype}
  \UseMicrotypeSet[protrusion]{basicmath} % disable protrusion for tt fonts
}{}
\makeatletter
\@ifundefined{KOMAClassName}{% if non-KOMA class
  \IfFileExists{parskip.sty}{%
    \usepackage{parskip}
  }{% else
    \setlength{\parindent}{0pt}
    \setlength{\parskip}{6pt plus 2pt minus 1pt}}
}{% if KOMA class
  \KOMAoptions{parskip=half}}
\makeatother
\usepackage{xcolor}
\usepackage[margin=1in]{geometry}
\usepackage{graphicx}
\makeatletter
\def\maxwidth{\ifdim\Gin@nat@width>\linewidth\linewidth\else\Gin@nat@width\fi}
\def\maxheight{\ifdim\Gin@nat@height>\textheight\textheight\else\Gin@nat@height\fi}
\makeatother
% Scale images if necessary, so that they will not overflow the page
% margins by default, and it is still possible to overwrite the defaults
% using explicit options in \includegraphics[width, height, ...]{}
\setkeys{Gin}{width=\maxwidth,height=\maxheight,keepaspectratio}
% Set default figure placement to htbp
\makeatletter
\def\fps@figure{htbp}
\makeatother
\setlength{\emergencystretch}{3em} % prevent overfull lines
\providecommand{\tightlist}{%
  \setlength{\itemsep}{0pt}\setlength{\parskip}{0pt}}
\setcounter{secnumdepth}{-\maxdimen} % remove section numbering
\ifLuaTeX
  \usepackage{selnolig}  % disable illegal ligatures
\fi
\IfFileExists{bookmark.sty}{\usepackage{bookmark}}{\usepackage{hyperref}}
\IfFileExists{xurl.sty}{\usepackage{xurl}}{} % add URL line breaks if available
\urlstyle{same}
\hypersetup{
  pdftitle={分析报告},
  pdfauthor={作者名},
  hidelinks,
  pdfcreator={LaTeX via pandoc}}

\title{分析报告}
\author{作者名}
\date{2024-02-26}

\begin{document}
\maketitle

\hypertarget{install-packages-required-for-the-analysis-uncomment-if-needed}{%
\section{Install packages required for the analysis (uncomment if
needed)}\label{install-packages-required-for-the-analysis-uncomment-if-needed}}

install.packages(``lmtest'') ; install.packages(``Epi'')
install.packages(``tsModel''); install.packages(``vcd'')

\hypertarget{load-the-packages}{%
\section{load the packages}\label{load-the-packages}}

library(foreign) library(tsModel) library(``lmtest'') library(``Epi'')
library(``splines'') library(``vcd'')

\hypertarget{read-data-from-csv-file}{%
\section{read data from csv file}\label{read-data-from-csv-file}}

data \textless- read.csv(``sicily.csv'') head(data) data

\hypertarget{descriptive-statistics}{%
\section{1. Descriptive Statistics}\label{descriptive-statistics}}

\hypertarget{compute-the-standardized-rates}{%
\section{compute the standardized
rates}\label{compute-the-standardized-rates}}

data\(rate <- with(data, aces/stdpop*10^5) # start the plot, excluding the points and the x-axis plot(data\)rate,type=``n'',ylim=c(00,300),xlab=``Year'',
ylab=``Std rate x 10,000'', bty=``l'',xaxt=``n'') \# shade the post
intervention period grey rect(36,0,60,300,col=grey(0.9),border=F) \#
plot the observed rate for pre-intervention period
points(data\(rate[data\)smokban==0{]},cex=0.7) \#specify the x-axis
(i.e.~time units) axis(1,at=0:5\emph{12,labels=F)
axis(1,at=0:4}12+6,tick=F,labels=2002:2006) \# add a title
title(``Sicily, 2002-2006'')

summary(data)

\#tabulate aces before and after the smoking ban
summary(data\(aces[data\)smokban==0{]})
summary(data\(aces[data\)smokban==1{]})
summary(data\(rate[data\)smokban==0{]})
summary(data\(rate[data\)smokban==1{]})

\hypertarget{regression-model}{%
\section{2. Regression model}\label{regression-model}}

\#Poisson with the standardised population as an offset model1
\textless- glm(aces \textasciitilde{} offset(log(stdpop)) + smokban +
time, family=poisson, data) summary(model1)
summary(model1)\(dispersion round(ci.lin(model1,Exp=T),3) datanew <- data.frame(stdpop=mean(data\)stdpop),smokban=rep(c(0,1),c(360,240)),
time= 1:600/10,month=rep(1:120/10,5)) \# We generate predicted values
based on the model in order to create a plot pred1 \textless-
predict(model1,type=``response'',datanew)/mean(data\(stdpop)*10^5 #This can then be plotted along with a scatter graph (see above) plot(data\)rate,type=``n'',ylim=c(0,300),xlab=``Year'',ylab=``Std
rate x 10,000'', bty=``l'',xaxt=``n'')
rect(36,0,60,300,col=grey(0.9),border=F)
points(data\(rate,cex=0.7) axis(1,at=0:5*12,labels=F) axis(1,at=0:4*12+6,tick=F,labels=2002:2006) lines((1:600/10),pred1,col=2) title("Sicily, 2002-2006") datanew <- data.frame(stdpop=mean(data\)stdpop),smokban=0,time=1:600/10,
month=rep(1:120/10,5)) \# generate predictions under the counterfactual
scenario and add it to the plot pred1b \textless-
predict(model1,datanew,type=``response'')/mean(data\(stdpop)*10^5 lines(datanew\)time,pred1b,col=2,lty=2)
\# return the data frame to the scenario including the intervention
datanew \textless-
data.frame(stdpop=mean(data\$stdpop),smokban=rep(c(0,1),c(360,240)),
time= 1:600/10,month=rep(1:120/10,5))

\#3. methodological issues

\#Model checking and autocorrelation \# Check the residuals by plotting
against time res2 \textless- residuals(model2,type=``deviance'')
plot(data\$time,res2,ylim=c(-5,10),pch=19,cex=0.7,col=grey(0.6),
main=``Residuals over time'',ylab=``Deviance residuals'',xlab=``Date'')
abline(h=0,lty=2,lwd=2) \# Further check for autocorrelation by
examining the autocorrelation and \# partial autocorrelation functions
acf(res2) pacf(res2)

\#4. adjusting for seasonality

\hypertarget{there-are-various-ways-of-adjusting-for-seasonality---here-we-use-harmonic}{%
\section{There are various ways of adjusting for seasonality - here we
use
harmonic}\label{there-are-various-ways-of-adjusting-for-seasonality---here-we-use-harmonic}}

\hypertarget{terms-specifying-the-number-of-sin-and-cosine-pairs-to-include-in-this}{%
\section{terms specifying the number of sin and cosine pairs to include
(in
this}\label{terms-specifying-the-number-of-sin-and-cosine-pairs-to-include-in-this}}

\hypertarget{case-2-and-the-length-of-the-period-12-months}{%
\section{case 2) and the length of the period (12
months)}\label{case-2-and-the-length-of-the-period-12-months}}

model3 \textless- glm(aces \textasciitilde{} offset(log(stdpop)) +
smokban + time + harmonic(month,2,12), family=quasipoisson, data)
summary(model3)
summary(model3)\(dispersion round(ci.lin(model3,Exp=T),3) # EFFECTS ci.lin(model3,Exp=T)["smokban",5:7] # TREND exp(coef(model3)["time"]*12) # We again check the model and autocorrelation functions res3 <- residuals(model3,type="deviance") plot(res3,ylim=c(-5,10),pch=19,cex=0.7,col=grey(0.6),main="Residuals over time",  ylab="Deviance residuals",xlab="Date") abline(h=0,lty=2,lwd=2) acf(res3) pacf(res3) # predict and plot of the seasonally adjusted model pred3 <- predict(model3,type="response",datanew)/mean(data\)stdpop)\emph{10\^{}5
plot(data\(rate,type="n",ylim=c(120,300),xlab="Year",ylab="Std rate x 10,000",  bty="l",xaxt="n") rect(36,120,60,300,col=grey(0.9),border=F) points(data\)rate,cex=0.7)
axis(1,at=0:5}12,labels=F)
axis(1,at=0:4\emph{12+6,tick=F,labels=2002:2006)
lines(1:600/10,pred3,col=2) title(``Sicily, 2002-2006'') \# it is
sometimes difficult to clearly see the change graphically in the \#
seasonally adjusted model, therefore it can be useful to plot a straight
\# line representing a `deseasonalised' trend \# this can be done by
predicting all the observations for the same month, in \# this case we
use June pred3b \textless-
predict(model3,type=``response'',transform(datanew,month=6))/
mean(data\$stdpop)}10\^{}5 \#this can then be added to the plot as a
dashed line lines(1:600/10,pred3b,col=2,lty=2)

\hypertarget{additional-material}{%
\section{5. Additional material}\label{additional-material}}

\hypertarget{add-a-change-in-slope}{%
\section{add a change-in-slope}\label{add-a-change-in-slope}}

\hypertarget{we-parameterize-it-as-an-interaction-between-time-and-the-ban-indicator}{%
\section{we parameterize it as an interaction between time and the ban
indicator}\label{we-parameterize-it-as-an-interaction-between-time-and-the-ban-indicator}}

model4 \textless- glm(aces \textasciitilde{} offset(log(stdpop)) +
smokban\emph{time + harmonic(month,2,12), family=quasipoisson, data)
summary(model4) round(ci.lin(model4,Exp=T),3) \# predict and plot the
`deseasonalised' trend \# compare it with the step-change only model
pred4b \textless-
predict(model4,type=``response'',transform(datanew,month=6))/
mean(data\(stdpop)*10^5 plot(data\)rate,type=``n'',ylim=c(120,300),xlab=``Year'',ylab=``Std
rate x 10,000'', bty=``l'',xaxt=``n'')
rect(36,120,60,300,col=grey(0.9),border=F) points(data\$rate,cex=0.7)
axis(1,at=0:5}12,labels=F) axis(1,at=0:4*12+6,tick=F,labels=2002:2006)
lines(1:600/10,pred3b,col=2) lines(1:600/10,pred4b,col=4)
title(``Sicily, 2002-2006'') legend(``topleft'',c(``Step-change
only'',``Step-change + change-in-slope''),lty=1,
col=c(2,4),inset=0.05,bty=``n'',cex=0.7) \# test if the change-in-slope
improve the fit \# the selected test here is an F-test, which accounts
for the overdispersion, \# while in other cases a likelihood ratio or
wald test can be applied anova(model3,model4,test=``F'') \# not
surprisingly, the p-value is similar to that of the interaction term

\end{document}
